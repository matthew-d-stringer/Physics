\documentclass[../Notes.tex]{subfiles}

% !TEX root = ../Notes.tex

\begin{document}
\section{Dynamics}

\subsection{Newton's First Law and Free Body diagrams}
\paragraph{Newton's First Law}
\begin{center}
\fbox{
	\begin{minipage}{30em}
	An object at rest will remain at rest, and an object in motion will remain in motion, at constant
	velocity and in a straight line, unless acted upon by a net force.
	\end{minipage}
}
\end{center}

\subsubsection{Force}
\begin{itemize}
	\item A force is a push or pull on an object.
	\item Units of force are in Newtons (N).
	\item A newton is roughly the weight of an apple
\end{itemize}
\begin{equation*}
1 N = 1 \frac{kg * m}{s^2}
\end{equation*}
\paragraph{Contract Force} \mbox{} \\
A force that arises that from direct contact between objects.
\begin{itemize}
	\item Tension
	\item Applied Force
	\item Friction
\end{itemize}
\paragraph{Field Force}\mbox{} \\
Forces that act at a distance.
\begin{itemize}
	\item Gravity
	\item Electrical
	\item Magnetic
\end{itemize}

\subsubsection{Net Force}
A net force is the vector sum of all the forces acting on an object.
\begin{equation*}
F_{net} = \sum F
\end{equation*}

\subsubsection{Equilibrium}
\begin{itemize}
	\item Static Equilibrium
	\begin{itemize}
		\item Net force is 0
		\item Net torque is 0
		\item Object is at rest
	\end{itemize}
	\item Mechanical Equilibrium
	\begin{itemize}
		\item Net force is 0
		\item Net torque is 0
	\end{itemize}
	\item Translational Equilibrium
	\begin{itemize}
		\item Net force is 0
	\end{itemize}
\end{itemize}

\subsubsection{Free Body Diagram}
A Free Body Diagram (FBD) is a diagram that maps all of the forces that are applied to a single object.

\subsection{Newton's 2nd and 3rd Laws of Motion}

\subsubsection{Newton's 2nd Law of Motion}
\begin{itemize}
	\item The acceleration of an object is in the direction of and directly proportional to the the
	net force applied, and inversely proportional to the object's mass.
	\item Valid only in \textit{inertial reference frames}.
\end{itemize}
\begin{equation*}
F_{net} = \sum F = ma
\end{equation*}

\subsubsection{Mass vs. Weight}
\begin{itemize}
	\item Mass is the amount of stuff that something is made up of (independent of gravity)
	\item Weight is the force of gravity on an object. (dependent on gravity)
\end{itemize}

\subsubsection{Newton's 3rd Law of Motion}
\begin{itemize}
	\item All forces come in pairs
	\item If Object 1 exerts a force on Object 2, then Object 2 must exert a force back on Object 2.
	\item This counter force is equal in magnitude and opposite in direction.
\end{itemize}
\begin{equation*}
F_{1on2} = -F_{2on1}
\end{equation*}

\subsection{Friction}

\subsubsection{Coefficient of Friction}
\begin{itemize}
	\item Ratio of the frictional force and the normal force
	\item 2 kinds:
	\begin{enumerate}
		\item Kinetic (when 2 objects are rubbing)
		\item Static (when 2 objects are not sliding)
	\end{enumerate}
\end{itemize}
\begin{equation*}
\mu = \frac{F_f}{F_N}
\end{equation*}
which results in 
\begin{equation*}
F_f = \mu F_N
\end{equation*}
where $F_f$ is the force of friction, $F_N$ is the normal force, and $\mu$ is the coefficient of 
friction.

\subsection{Retarding or Drag Forces}

\subsubsection{Retarding Forces}
\begin{itemize}
	\item Frictional forces can be functions of the object's velocity
	\item These forces are called drag or retarding forces
\end{itemize}

\subsubsection{The Skydiver}
Typically drag forces on a free-falling object take the form of 
\begin{equation*}
F_{drag} = bv
\end{equation*}
or
\begin{equation*}
F_{drag} = cv^2
\end{equation*}
By using Newton's 2nd Law, create a differential equation. Then use separation of variables to solve
for velocity, then acceleration, then position.

\subsection{Ramps and Inclines}

\subsubsection{Drawing FBD for Ramps}
\begin{enumerate}
	\item Choose the object and draw it as a dot or box
	\item Draw and Label all the External Forces
	\item Sketch a Coordinate System
\end{enumerate}

\subsection{Atwood Machine}

\subsubsection{What is an Atwood Machine}
Two objects connected by a light string over a massless pulley

\subsubsection{Properties of Atwood Machines}
\begin{itemize}
	\item Ideal pulleys are frictionless and massless
	\item Tension is constant in a light string passing over an ideal pulley
\end{itemize}

\subsubsection{Setup for Atwood Machines}
\begin{enumerate}
	\item Adopt a sign convention for positive and negative motion
	\item Analyze each mass separately using Newton's 2nd Law.
\end{enumerate}

\subsubsection{Solution}
\begin{align*}
F_{y} = m_1 g - m_2 g = (m_1 + m_2)a \\
(m_1 - m_2)g = (m_1 + m_2)a \\
a = g \frac{(m_1 - m_2)}{(m_1 + m_2)}
\end{align*}

\end{document}