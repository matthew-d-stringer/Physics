\documentclass[../Notes.tex]{subfiles}

% !TEX root = ../Notes.tex

\begin{document}

\section{Kinematics}
\subsection{Describing Motion 1}

\subsubsection{Average Speed}
\begin{itemize}
	\item Average speed is the distance traveled over change in time
	\item It is a scaler
	\item Measured in meters/second.
	\item Magnitude of Velocity Vector
\end{itemize}

\subsubsection{Average Velocity}
\begin{itemize}
	\item Average velocity is a vector.
	\item Measured in meters/second.
\end{itemize}
\begin{equation*}
v_{avg} = \frac{\Delta x}{\Delta t} = \frac{x_f - x_i}{t_f - t_i}
\end{equation*}

\subsubsection{Average Velocity}
\begin{itemize}
	\item Rate that velocity changes
	\item Is a vector
	\item Units are meters/second/second
\end{itemize}
\begin{equation*}
a = \frac{\Delta v}{\Delta t} = \frac{dv}{dt}
\end{equation*}

\subsubsection{Displacement}
The displacement from $t_0$ to $t_1$ of a position function $x(t)$ with velocity function $v(t)$ is
\begin{equation*} 
\int_{t_0}^{t_1} v(t)dt
\end{equation*} 

\newpage
\subsection{Describing Motion 2}

\subsubsection{Kinematic Equations}
\begin{center}
\textbf{Variables}\\
\begin{tabular}{|c|c|}
\hline 
$v_0$ & Initial velocity \\
$v$ & Final velocity \\
$\Delta x$ & Displacement \\
$a$ & Acceleration \\
$t$ & Time \\
\hline
\end{tabular}
\end{center}
\begin{itemize}
	\item $v = v_0 + at$
	\item $x = x_0 + v_0 t + \frac12 at^2$
	\item $v^2 = v_0^2 + 2a\Delta x$
\end{itemize}

\subsubsection{Acceleration Due to Gravity}
\begin{itemize}
	\item Near the surface of Earth, objects accelerate at a rate of $9.8 \frac m{s^2}$
	\item This is acceleration due to gravity ($g$)
	\item This can be approximated to $10 \frac m{s^2}$
	\item As you move from Earth, acceleration decreases.
\end{itemize}

\subsubsection{Objects Falling From Rest}
\begin{itemize}
	\item Objects starting from rest have $v_0 = 0$
	\item Typically down is the positive direction
	\item Acceleration is $+g$.
\end{itemize}

\subsubsection{Objects Launched Upward}
\begin{itemize}
	\item Must examine the motion of object going up and down.
	\item Since object is going up, that is the positive direction.
	\item Acceleration is $-g$.
	\item At the highest point, $v = 0$.
\end{itemize}

\newpage
\subsection{Projectile Motion}
A \textbf{projectile} is an object that is acted upon only gravity.

\subsubsection{Independence of Motion}
\begin{itemize}
	\item Projectiles launched at an angle have motion in 2 dimensions.
	\begin{itemize}
		\item Vertical - acceleration is gravity
		\item Horizontal - 0 acceleration
	\end{itemize}
	\item Vertical and Horizontal motion are treated separately
\end{itemize}

\begin{center}
\fbox{\begin{minipage}{30em}
\textbf{Note:} An object will travel the maximum horizontal distance with a launch angle of $45 \degree$
\end{minipage}}
\end{center}

\subsubsection{Steps for any Projectile Motion Problem}
\begin{enumerate}
	\item First, know that 
	\begin{equation*}
	a = 
	\begin{bmatrix}
	0 \\ -g
	\end{bmatrix}
	\end{equation*}
	\item Then find your $v_0$ as a vector 
	\item Find your $x_0$ as a vector
	\item Substitute your vectors into the following formula
	\begin{equation*}
		x(t) = -\frac12 at^2 + v_0 t + x_0
	\end{equation*}
\end{enumerate}

\subsubsection{Graphing Projectile Motion}
In order to graph a path, solve for $y = f(x)$. Do this by solving for $t$ in relation to $x$ and 
then substitute into the y component. \\
For example: 
\begin{align*}
x = f(t) \\
y = g(t) \\
\text{Find } y = h(x) \\
t = f^{-1}(x) \\
y = g(f^{-1}(x)) = h(x) \\
\text{so } h(x) = g(f^{-1}(x))
\end{align*}

\subsection{Circular And Relative Motion}
\subsubsection{Converting Linear to Angular Velocity} \mbox{}\\
If we have an object moving counter clockwise around a point, let $\omega = \frac{d \theta}{dt}$. If
we know that the object has velocity $v$ and position $s$, we know that $s=r\theta$ where $r$ is the
radius of the circular path. By taking the derivative of both sides, $\dot{s} = r \dot{\theta}$. 
Now we can substitute to find the angular velocity.
\begin{equation*}
\dot{s} = r \omega
\end{equation*}
\end{document}