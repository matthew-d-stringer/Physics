\documentclass[../Notes.tex]{subfiles}

% !TEX root = ../Notes.tex

\begin{document}

\section{Work, Energy, and Power}

\subsection{Work}

\subsubsection{What is Work}
\begin{itemize}
	\item Work is the process of moving an object by applying a force
	\item The object must move
	\item The force must cause the movement
	\item Work is measured in Joules
\end{itemize}
\begin{equation*}
W = F \cdot \Delta x = F \Delta x \cos \theta 
\end{equation*}

\subsubsection{Non-Constant Forces}
\begin{itemize}
	\item Work done is the area under the force vs. displacement graph.
\end{itemize}
\begin{equation*}
W = \int_{x_i}^{x_f} F(x)dx
\end{equation*}

\subsubsection{Hook's Law}
\begin{itemize}
	\item The more you stretch or compress a spring, the greater the force of the spring.
	\item The spring's force is opposite the direction of its displacement from equilibrium.
	\item This is modeled as a linear relationship.
\end{itemize}
\begin{equation*}
F_s = -k x
\end{equation*}

\subsubsection{Determining the Spring Constant}
\begin{itemize}
	\item Graph Force vs Displacement for the Spring
	\item The slope of this graph is the Spring Constant
\end{itemize}
\begin{equation*}
k = \frac{\Delta F}{\Delta x}
\end{equation*}

\subsubsection{Work in Multiple Dimensions}
\begin{align*}
W = \int dW \\
W = \int_{r_1}^{r_2} F \cdot dr 
\end{align*}

\subsubsection{Work-Energy Theorem}
\begin{align*}
W = \int_{x_i}^{x_f} F(x) dx \\
F = ma = m \frac{dv}{dt} \qquad v = \frac{dx}{dt} \quad dx = vdt \\
W = \int m \frac{dv}{dt} v \; dt \\
W = \int_{v_i}^{v_f} m v \; dv  \\
{\bf W = m \int_{v_i}^{v_f} v \; dv} \\
= m \left. \frac{v^2}{2} \right|_{v_i}^{v_f} \\
= m (\frac{v_f^2 - v_i^2}{2}) \\
K = \frac12 mv^2 \quad \text{$K$ is kinetic energy} \\
W = K_f - K_i = \Delta K
\end{align*}
Energy Formula:
\begin{equation*}
	{\bf K = \frac{1}{2} mv^2}
\end{equation*}

\subsection{Energy and Conservative Forces}

\subsubsection{What is Energy?}
\begin{itemize}
	\item Energy is the ability to do work
	\item in other words, Energy is the ability to move an object
\end{itemize}

\subsubsection{Kinetic Energy}
\begin{itemize}
	\item Kinetic Energy is energy of motion.
	\begin{itemize}
		\item The ability or capacity of moving object move another object.
	\end{itemize}
\end{itemize}
\begin{equation*}
	K = \frac12 mv^2
\end{equation*}

\subsubsection{Potential Energy}
\begin{itemize}
	\item Potential Energy (U) is energy an object possesses due to its position or state of being.
	\item A single object can only have kinetic energy, because potential energy requires an 
		interaction between objects.
\end{itemize}

\subsubsection{Internal Energy}
\begin{itemize}
	\item The internal energy of a system includes the kinetic energy and potential energy.
	\item By changing a system's internal structure, you can change its internal energy.
\end{itemize}

\subsubsection{Gravitational Potential Energy ($U_g$)}
\begin{equation*}
	U_g = mg \Delta h
\end{equation*}

\subsubsection{Conservative Forces}
\begin{itemize}
	\item A force in which the work done on an object is independent of the path taken
	\item A force in which the work done moving along a closed path is zero
	\item A force in which the work done is directly related to the change in potential energy
	\begin{equation*}
		W = - \Delta U
	\end{equation*}
\end{itemize}
Conservative Forces
\begin{itemize}
	\item Gravity
	\item Elastic Forces
	\item Coulumbic
\end{itemize}
Non-Conservative Forces
\begin{itemize}
	\item Friction
	\item Drag
	\item Air Resistance
\end{itemize}

\subsubsection{Work Done by Conservative Forces}
\begin{align*}
	W = -\Delta U \implies 
	\Delta U = -W \\
	\Delta U = - \int_{r_i}^{r_f} F \cdot dr
\end{align*}

\subsubsection{Newton's Law of Universal Gravitation}
\begin{equation*}
	F_g = \frac{-G m_1 m_2}{r^2} {\hat r}
\end{equation*}

\subsubsection{Gravitational Potential Energy}
\begin{align*}
	U_g &= - \int_{\infty}^{r} F_g \cdot dr \\
	U_g &= G m_1 m_2 \int_{\infty}^{r} r^{-2} dr \\
	U_g &= G m_1 m_2 \left[ - r^{-1} \right]_{\infty}^{r} \\
	U_g &= G m_1 m_2 \left( - r^{-1} - 0 \right) \\
	U_g &= -\frac{G m_1 m_2}{r} 
\end{align*}

\subsubsection{Elastic Potential Energy}
\begin{align*}
	U_s &= - \int_{0}^{x} F_s \cdot dx \\
	U_s &= - \int_{0}^{x} -kx dx \\
	U_s &= k \left[ \frac{x^2}2 \right]_0^x \\
	U_s &= \frac{kx^2}{2}
\end{align*}

\subsubsection{Force from Potential Energy}
\begin{align*}
	dU = -dW_f &= - F \cdot dl \\
	&= -F \cos \theta dl \\
	&= -F_l dl \\
	F_l = -\frac{dU}{dl}
\end{align*}
Where $F_l$ is the force in the direction of potential energy

\subsubsection{Gravitational Force from the Gravitational Potential Energy}
\begin{align*}
	F_r &= -\frac{dU}{dr} \\
	F_r &= \frac{d}{dr} \frac{G m_1 m_2}{r} \\
	F_r &= G m_1 m_2 \frac{d}{dr} r^{-1} \\
	F_r &= G m_1 m_2 \frac{-1}{r^2} \\
	F_r &= - \frac{G m_1 m_2}{r^2} \\
	F &= - \frac{G m_1 m_2}{r^2} \hat r
\end{align*}

\subsection{Conservation of Energy}

\subsubsection{Conservation of Mechanical Energy}
\begin{align*}
	W_f = \Delta K \quad W_f = -\Delta U \\
	\Delta K = - \Delta U \\
	\Delta K + \Delta U = 0
\end{align*}

\subsubsection{Non-Conservative Forces}
\begin{itemize}
    \item change total mechanical energy of a system 
    \item Work done is typically converted to internal (thermal) energy.
\end{itemize}
\begin{align*}
    E_{TOTAL} = K + U + W_{NC} \\ 
    E_{MECHANICAL} = K + U
\end{align*}
Where $W_{NC}$ is work done by non-conservative forces

\subsection{Power}

\subsubsection{Definition}
\begin{itemize}
	\item Power is the rate at which work is done.
	\item Units of power are joules/second, or watts
	\item Average power:
	\begin{equation*}
		P_{avg} = \frac{\Delta W}{\Delta t}
	\end{equation*}
\end{itemize}

\subsubsection{Instantaneous Power}
\begin{align*}
	P &= \frac{dW}{dt} \\
	&= \frac{Fdr}{dt} \\
	&= F \cdot v
\end{align*}

\end{document}