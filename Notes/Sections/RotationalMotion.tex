\documentclass[../Notes.tex]{subfiles}

% !TEX root = ../Notes.tex

\begin{document}
    \section{Rotational Motion}

    \subsection{Rotational Kinematics}

    \subsubsection{Linear vs Angular Velocity}
    \begin{itemize}
        \item Linear speed given by $v$
        \item Angular speed given by $\omega$
        \item Direction is perpendicular to the path based on right hand rule
    \end{itemize}

    \subsubsection{Converting Linear to angular velocity}
    \begin{equation*}
        v = r \frac{d\theta}{dt} = r \omega
    \end{equation*}
    \begin{equation*}
        \omega = \frac{v}{r}
    \end{equation*}

    \subsubsection{Linear vs Angular acceleration}
    \begin{itemize}
        \item Linear acceleration given by $a$
        \item Angular acceleration given by $\alpha$
    \end{itemize}

    \subsubsection{Centripetal Acceleration}
    \begin{equation*}
        a = -\frac{v^2}{r}
    \end{equation*}

    \subsection{Moment of Inertia}
    \subsubsection{Types of Inertia}
    \begin{itemize}
        \item Inertial mass (linear inertia) is an object's ability to resist 
            linear acceleration.
        \item Moment of Inertia (rotational inertia) is an object's ability to resist 
            rotational acceleration.
    \end{itemize}

    \subsubsection{Kinetic Energy of Rotating Disk}
    \begin{equation*}
        K_{TOT} = \frac{\omega^2}{2} \int_0^r r^2 dr = \frac12 J \omega^2
    \end{equation*}

    \subsubsection{Common Moments of Inertia}
    \begin{itemize}
        \item Disk: $J = \frac12 mr^2$
        \item Hoop: $J = ml^2$ 
        \item Sphere: $J = \frac25 m r^2$
        \item Hollow Sphere: $J = \frac23 m r^2$
        \item Rod(around center point): $J = \frac1{12} m l^2$
        \item Rod(around end point): $J = \frac13 m l^2$
    \end{itemize}

    \subsection{Torque}
    \begin{align*}
        \tau = r \times F \\
        |\tau| = rF\sin \theta
    \end{align*}
    \subsubsection{Direction of Torque Vector}
    \begin{itemize}
        \item Torque vector is perpendicular to both force and position vector
        \item Use the right hand rule
        \item Positive Torques cause counter-clockwise rotations 
    \end{itemize}
    \subsubsection{Translational vs Rotational}
    \begin{align*}
        F = m a \quad
        \tau = J \alpha
    \end{align*}

    \subsection{Angular Momentum ($L$)}
    \begin{equation*}
        L_Q = r \times p = (r \times v)m
    \end{equation*}
    where $r$ is a vector from point $Q$ to the force
\end{document}