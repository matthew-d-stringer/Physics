\documentclass[12pt]{article}
\usepackage{tikz}
\usetikzlibrary{matrix,arrows}
\usepackage{graphicx}
\usepackage{amsmath} 
\usepackage{amssymb} 
\usepackage{amsthm}
\usepackage{enumitem} % to change appearance of enum and item environments
\usepackage{framed}
\usepackage{color}
\usepackage{multicol}
\usepackage{gensymb}
%\usepackage{yhmath}
% adjust the margins using the geometry package
\usepackage[left=0.75in, right=0.75in, top=0.75in, bottom=0.75in]{geometry}
\usepackage[parfill]{parskip}
 %\usepackage{mathpazo}
%\usepackage{euler}

% customize the headers using the fancyhdr package
\usepackage{fancyhdr}
\pagestyle{fancy}

\usepackage{hyperref}

\usepackage{mathrsfs}
\usepackage{mathtools}

\renewcommand{\headrulewidth}{0.4pt}
\renewcommand{\footrulewidth}{0.4pt}

\newenvironment{hwproblem}[1]{{\large \bfseries Problem #1\,:} \begin{trivlist}\item[]\vspace{-0.5ex}}{\end{trivlist}\vspace{3ex}}
\newenvironment{hwproblemquestion}[1]{{\bfseries Question #1\,:} \begin{trivlist}\item[]\vspace{-1.5ex}}{\end{trivlist}\vspace{3ex}}
\newenvironment{lecexercise}[1]{{\large \bfseries Exercise #1\,:} \begin{trivlist}\item[]\vspace{-0.5ex}}{\end{trivlist}\vspace{3ex}}
\newenvironment{note}[1]{{\large \bfseries Note #1\,:} \begin{trivlist}\item[]\vspace{-0.5ex}}{\end{trivlist}\vspace{3ex}}
\newenvironment{hwsubpart}[1]{{\bfseries #1\,:} \begin{trivlist}\item[]\vspace{-0.5ex}}{\end{trivlist}\vspace{3ex}}


% create theorem style
\newtheoremstyle{break}% name
  {}%         Space above, empty = `usual value'
  {}%         Space below
  {}%         Body font
  {}%         Indent amount (empty = no indent, \parindent = para indent)
  {\bfseries}% Thm head font
  {.}%        Punctuation after thm head
  {\newline}% Space after thm head: \newline = linebreak
  {}%         Thm head spec

\theoremstyle{break}
\newtheorem{hwquestion}{Question}

\newtheorem{theorem}{Theorem}
\numberwithin{theorem}{subsection}

\newtheorem*{lemma*}{Lemma}
\newtheorem{lemma}{Lemma}
\numberwithin{lemma}{subsection}

\newtheorem{corollary}{Corollary}
\numberwithin{corollary}{subsection}


\newtheorem*{definition}{Definition}

\newtheorem*{note*}{Note}

\newtheorem*{remark}{Remark}

\newtheorem*{hint}{Hint}

\newtheorem*{example}{Example}

\numberwithin{equation}{subsection}



% customize enumerate and itemize environments
\setlist[itemize]{labelsep=1ex,itemsep=1.5ex,parsep=0ex,leftmargin=4ex,topsep=0.5ex}
\setlist[enumerate]{labelsep=1ex,itemsep=1.5ex,parsep=0ex,leftmargin=4ex,topsep=0.5ex}




\rhead{AP Physics}
\lhead{Matthew Stringer} % your name
\lfoot{}
\cfoot{AP Physics Notes}  % change to the corresponding number
\rfoot{\thepage}

\setlength{\headheight}{15pt}
\title{Physics Notes} % change to the corresponding number
\date{}
\author{Matthew Stringer} % your name and ID number

% Anything above the \begin{document} is the template. If you wish to start a new document using this template, erase everything inside of the \begin{document}...\end{document}
\allowdisplaybreaks
\begin{document}
\maketitle
\newpage
\tableofcontents

%\newpage
%\part{AP Physics 1}

\newpage
\part{Physics C: Mechanics}

\section{Kinematics}
\subsection{Describing Motion 1}

\subsubsection{Average Speed}
\begin{itemize}
	\item Average speed is the distance traveled over change in time
	\item It is a scaler
	\item Measured in meters/second.
	\item Magnitude of Velocity Vector
\end{itemize}

\subsubsection{Average Velocity}
\begin{itemize}
	\item Average velocity is a vector.
	\item Measured in meters/second.
\end{itemize}
\begin{equation*}
v_{avg} = \frac{\delta x}{\delta t} = \frac{x_f - x_i}{t_f - t_i}
\end{equation*}

\subsubsection{Average Velocity}
\begin{itemize}
	\item Rate that velocity changes
	\item Is a vector
	\item Units are meters/second/second
\end{itemize}
\begin{equation*}
a = \frac{\delta v}{\delta t} = \frac{dv}{dt}
\end{equation*}

\subsubsection{Displacement}
The displacement from $t_0$ to $t_1$ of a position function $x(t)$ with velocity function $v(t)$ is
\begin{equation*} 
\int_{t_0}^{t_1} v(t)dt
\end{equation*} 

\newpage
\subsection{Describing Motion 2}

\subsubsection{Kinematic Equations}
\begin{center}
\textbf{Variables}\\
\begin{tabular}{|c|c|}
\hline 
$v_0$ & Initial velocity \\
$v$ & Final velocity \\
$\delta x$ & Displacement \\
$a$ & Acceleration \\
$t$ & Time \\
\hline
\end{tabular}
\end{center}
\begin{itemize}
	\item $v = v_0 + at$
	\item $x = x_0 + v_0 t + \frac12 at^2$
	\item $v^2 = v_0^2 + 2a\delta x$
\end{itemize}

\subsubsection{Acceleration Due to Gravity}
\begin{itemize}
	\item Near the surface of Earth, objects accelerate at a rate of $9.8 \frac m{s^2}$
	\item This is acceleration due to gravity ($g$)
	\item This can be approximated to $10 \frac m{s^2}$
	\item As you move from Earth, acceleration decreases.
\end{itemize}

\subsubsection{Objects Falling From Rest}
\begin{itemize}
	\item Objects starting from rest have $v_0 = 0$
	\item Typically down is the positive direction
	\item Acceleration is $+g$.
\end{itemize}

\subsubsection{Objects Launched Upward}
\begin{itemize}
	\item Must examine the motion of object going up and down.
	\item Since object is going up, that is the positive direction.
	\item Acceleration is $-g$.
	\item At the highest point, $v = 0$.
\end{itemize}

\newpage
\subsection{Projectile Motion}
A \textbf{projectile} is an object that is acted upon only gravity.

\subsubsection{Independence of Motion}
\begin{itemize}
	\item Projectiles launched at an angle have motion in 2 dimensions.
	\begin{itemize}
		\item Vertical - acceleration is gravity
		\item Horizontal - 0 acceleration
	\end{itemize}
	\item Vertical and Horizontal motion are treated separately
\end{itemize}

\begin{center}
\fbox{\begin{minipage}{30em}
\textbf{Note:} An object will travel the maximum horizontal distance with a launch angle of $45 \degree$
\end{minipage}}
\end{center}

\subsubsection{Steps for any Projectile Motion Problem}
\begin{enumerate}
	\item First, know that 
	\begin{equation*}
	a = 
	\begin{bmatrix}
	0 \\ -g
	\end{bmatrix}
	\end{equation*}
	\item Then find your $v_0$ as a vector 
	\item Find your $x_0$ as a vector
	\item Substitute your vectors into the following formula
	\begin{equation*}
		x(t) = -\frac12 at^2 + v_0 t + x_0
	\end{equation*}
\end{enumerate}

\subsubsection{Graphing Projectile Motion}
In order to graph a path, solve for $y = f(x)$. Do this by solving for $t$ in relation to $x$ and 
then substitute into the y component. \\
For example: 
\begin{align*}
x = f(t) \\
y = g(t) \\
\text{Find } y = h(x) \\
t = f^{-1}(x) \\
y = g(f^{-1}(x)) = h(x) \\
\text{so } h(x) = g(f^{-1}(x))
\end{align*}

\subsection{Circular And Relative Motion}
\subsubsection{Converting Linear to Angular Velocity} \mbox{}\\
If we have an object moving counter clockwise around a point, let $\omega = \frac{d \theta}{dt}$. If
we know that the object has velocity $v$ and position $s$, we know that $s=r\theta$ where $r$ is the
radius of the circular path. By taking the derivative of both sides, $\dot{s} = r \dot{\theta}$. 
Now we can substitute to find the angular velocity.
\begin{equation*}
\dot{s} = r \omega
\end{equation*}

\section{Dynamics}

\subsection{Newton's First Law and Free Body diagrams}
\paragraph{Newton's First Law}
\begin{center}
\fbox{
	\begin{minipage}{30em}
	An object at rest will remain at rest, and an object in motion will remain in motion, at constant
	velocity and in a straight line, unless acted upon by a net force.
	\end{minipage}
}
\end{center}

\subsubsection{Force}
\begin{itemize}
	\item A force is a push or pull on an object.
	\item Units of force are in Newtons (N).
	\item A newton is roughly the weight of an apple
\end{itemize}
\begin{equation*}
1 N = 1 \frac{kg * m}{s^2}
\end{equation*}
\paragraph{Contract Force} \mbox{} \\
A force that arises that from direct contact between objects.
\begin{itemize}
	\item Tension
	\item Applied Force
	\item Friction
\end{itemize}
\paragraph{Field Force}\mbox{} \\
Forces that act at a distance.
\begin{itemize}
	\item Gravity
	\item Electrical
	\item Magnetic
\end{itemize}

\subsubsection{Net Force}
A net force is the vector sum of all the forces acting on an object.
\begin{equation*}
F_{net} = \sum F
\end{equation*}

\subsubsection{Equilibrium}
\begin{itemize}
	\item Static Equilibrium
	\begin{itemize}
		\item Net force is 0
		\item Net torque is 0
		\item Object is at rest
	\end{itemize}
	\item Mechanical Equilibrium
	\begin{itemize}
		\item Net force is 0
		\item Net torque is 0
	\end{itemize}
	\item Translational Equilibrium
	\begin{itemize}
		\item Net force is 0
	\end{itemize}
\end{itemize}

\subsubsection{Free Body Diagram}
A Free Body Diagram (FBD) is a diagram that maps all of the forces that are applied to a single object.

\subsection{Newton's 2nd and 3rd Laws of Motion}

\subsubsection{Newton's 2nd Law of Motion}
\begin{itemize}
	\item The acceleration of an object is in the direction of and directly proportional to the the
	net force applied, and inversely proportional to the object's mass.
	\item Valid only in \textit{inertial reference frames}.
\end{itemize}
\begin{equation*}
F_{net} = \sum F = ma
\end{equation*}

\subsubsection{Mass vs. Weight}
\begin{itemize}
	\item Mass is the amount of stuff that something is made up of (independent of gravity)
	\item Weight is the force of gravity on an object. (dependent on gravity)
\end{itemize}

\subsubsection{Newton's 3rd Law of Motion}
\begin{itemize}
	\item All forces come in pairs
	\item If Object 1 exerts a force on Object 2, then Object 2 must exert a force back on Object 2.
	\item This counter force is equal in magnitude and opposite in direction.
\end{itemize}
\begin{equation*}
F_{1on2} = -F_{2on1}
\end{equation*}

\subsection{Friction}

\subsubsection{Coefficient of Friction}
\begin{itemize}
	\item Ratio of the frictional force and the normal force
	\item 2 kinds:
	\begin{enumerate}
		\item Kinetic (when 2 objects are rubbing)
		\item Static (when 2 objects are not sliding)
	\end{enumerate}
\end{itemize}
\begin{equation*}
\mu = \frac{F_f}{F_N}
\end{equation*}
which results in 
\begin{equation*}
F_f = \mu F_N
\end{equation*}
where $F_f$ is the force of friction, $F_N$ is the normal force, and $\mu$ is the coefficient of 
friction.

\subsection{Retarding or Drag Forces}

\subsubsection{Retarding Forces}
\begin{itemize}
	\item Frictional forces can be functions of the object's velocity
	\item These forces are called drag or retarding forces
\end{itemize}

\subsubsection{The Skydiver}
Typically drag forces on a free-falling object take the form of 
\begin{equation*}
F_{drag} = bv
\end{equation*}
or
\begin{equation*}
F_{drag} = cv^2
\end{equation*}
By using Newton's 2nd Law, create a differential equation. Then use separation of variables to solve
for velocity, then acceleration, then position.

\subsection{Ramps and Inclines}

\subsubsection{Drawing FBD for Ramps}
\begin{enumerate}
	\item Choose the object and draw it as a dot or box
	\item Draw and Label all the External Forces
	\item Sketch a Coordinate System
\end{enumerate}

\subsection{Atwood Machine}

\subsubsection{What is an Atwood Machine}
Two objects connected by a light string over a massless pulley

\subsubsection{Properties of Atwood Machines}
\begin{itemize}
	\item Ideal pulleys are frictionless and massless
	\item Tension is constant in a light string passing over an ideal pulley
\end{itemize}

\subsubsection{Setup for Atwood Machines}
\begin{enumerate}
	\item Adopt a sign convention for positive and negative motion
	\item Analyze each mass separately using Newton's 2nd Law.
\end{enumerate}

\subsubsection{Solution}
\begin{align*}
F_{y} = m_1 g - m_2 g = (m_1 + m_2)a \\
(m_1 - m_2)g = (m_1 + m_2)a \\
a = \frac{(m_1 - m_2)}{(m_1 + m_2)}
\end{align*}

\end{document}
